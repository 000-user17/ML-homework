%!TEX program = xelatex
\documentclass[a4paper,UTF8]{article}

\usepackage{amssymb}
\usepackage{amsmath}
\usepackage{amsthm}
\usepackage{bm}
\usepackage{color}
\usepackage{ctex}
\usepackage{enumerate}
\usepackage{epsfig}
\usepackage[margin=1.25in]{geometry}
\usepackage{graphicx}
\usepackage{hyperref}
\usepackage{lipsum}
\usepackage{mdframed}
\usepackage{pgfplots}
\usepackage{tcolorbox}

\setlength{\evensidemargin}{.25in}
\setlength{\textwidth}{6in}
\setlength{\topmargin}{-0.5in}
\setlength{\topmargin}{-0.5in}

%%%%%%%%%%%%%%%%%%此处用于设置页眉页脚%%%%%%%%%%%%%%%%%%
\usepackage{fancyhdr}
\usepackage{lastpage}
\usepackage{layout}
\footskip = 12pt 
\pagestyle{fancy}
\lhead{2022年春季}                    
\chead{机器学习理论研究导引}
\rhead{作业三}
\cfoot{\thepage}                                                
\renewcommand{\headrulewidth}{1pt} %页眉线宽, 设为0可以去页眉线
\setlength{\skip\footins}{0.5cm} %脚注与正文的距离           
\renewcommand{\footrulewidth}{0pt} %页脚线宽, 设为0可以去页脚线

\makeatletter %设置双线页眉                                        
\def\headrule{{\if@fancyplain\let\headrulewidth\plainheadrulewidth\fi%
		\hrule\@height 1.0pt \@width\headwidth\vskip1pt	%上面线为1pt粗  
		\hrule\@height 0.5pt\@width\headwidth %下面0.5pt粗            
		\vskip-2\headrulewidth\vskip-1pt} %两条线的距离1pt        
	\vspace{6mm}} %双线与下面正文之间的垂直间距              
\makeatother  

%%%%%%%%%%%%%%%%%%%%%%%%%%%%%%%%%%%%%%%%%%%%%%
\numberwithin{equation}{section}
\newtheorem*{myProof}{Proof}
\newtheorem*{mySol}{Solution}

\usepackage{multirow}

\def \x {\boldsymbol{x}}
\def \L {\mathfrak{L}}
\def \z {\boldsymbol{z}}
\def \w {\boldsymbol{w}}
\def \E {\mathbb{E}}
\def \D {\mathcal{D}}
\newcommand\norm[1]{\left\lVert #1 \right\rVert}

\pgfplotsset{compat=1.16} 

\begin{document}
	
	\title{机器学习理论研究导引\\
		作业三}
	\author{陈晟\, MG21330006} 
	\maketitle
	%%%%%%%% 注意: 使用XeLatex 编译可能会报错, 请使用 pdfLaTex 编译 %%%%%%%
	
	\section*{作业提交注意事项}
	
	\begin{tcolorbox}
		\begin{enumerate}
			\item[(1)] 本次作业提交截止时间为~\textcolor{red}{\textbf{2022/05/03  23:59:59}}, 截止时间后不再接收作业, 本次作业记零分; 
			\item[(2)] 作业提交方式:使用此~LaTex~模板书写解答,只需提交编译生成的~pdf~文件,将~pdf~文件上传到以下ftp服务器的指定位置:
			\newline 地址:sftp://210.28.132.67:22,用户名:mlt2022,密码:mltspring2022@nju
			\newline 文件夹位置:/C:/Users/mlt2022/hw\_submissions/hw3\_submission/  ;
			\item[(3)] pdf 文件命名方式:学号-姓名-作业号-v版本号, 例~ MG1900000-张三-3-v1;如果需要更改已提交的解答,请在截止时间之前提交新版本的解答,并将版本号加一;
			\item[(4)] 未按照要求提交作业,或~pdf~命名方式不正确,将会被扣除部分作业分数. 
		\end{enumerate}
	\end{tcolorbox}
	
	\newpage
	
	\section{[50pts] Generalization Bound Based-on VC Dimension} 
	在书中,为了推导基于VC维的泛化界,先推导出了基于增长函数的泛化界,再利用VC维和增长函数之间的关系(定理3.1)完成证明。在本题中,我们将从基于Rademacher复杂度的泛化界出发,推导基于VC维的泛化界。若假设空间$\mathcal{H}$的VC维为$d$,
	\begin{enumerate}
		\item[(1)]\textbf{[30pts]} 试证明对特征空间上的任一分布$\mathcal{D}$和任一正整数$m$都有
		$$ \mathfrak{R}_m(\mathcal{H})\leq\sqrt{\frac{2d\ln\left(\frac{em}{d}\right)}{m}}~. $$
		\item[(2)]\textbf{[20pts]} 试利用基于Rademacher复杂度的泛化界,结合(1)中的结果,推导基于VC维的泛化界。
	\end{enumerate}
	
	\begin{myProof}~\\ 
		
		\textbf{(1)} 
		
		VC维:$VCdim(\mathcal{H} = max\left \{ m:\prod _{\mathcal{H}}(m)=2^m \right \} $
		
		Let $\mathcal{H}$ be a hypothesis set with $VCdim(\mathcal{H})$.Then for all $m \geq d$,
		
		$\prod _{H}(m) \leq \sum_{i=0}^{d}\binom{m}{i} \leq \sum_{i=0}^{d} \binom{m}{i}(\frac{m}{d} )^{d-i}$
		
		$\leq \sum_{m}^{i=0}\binom{m}{i}(\frac{m}{d} )^{d-i}=(\frac{m}{d} )^d\sum_{i=0}^{m}\binom{m}{i}  (\frac{d}{m} )^i$
		
		$=\left ( \frac{m}{d}  \right )^d(1+\frac{d}{m})^m \leq (\frac{m}{d} )^de^d  $
		
		$\therefore  {\textstyle \prod_{\mathcal{H}}}(m) \leq (\frac{em}{d} )^d = O(m^d) $
		
		对于$D = \left \{ x_1, x_2... x_m \right \} $, $\mathcal{H_{|D}}$为假设空间$\mathcal{H}$在D上的限制。由于$h\in \mathcal{H}$的值域为$\left \{ -1,+1 \right \} $,可知$\mathcal{H_{|D}}$中的元素为模长$\sqrt{m} $的向量。因此由ppt中定理可得:
		
		$\because \mathcal{R}_m(\mathcal{H}) = E_D[E_\sigma [sup_{u\in \mathcal{H}|D}\frac{1}{m}\sum_{m}^{i=1}\sigma _iu_i  ]]$
		
		$\leq E_D[\frac{\sqrt{m}\sqrt{2ln|\mathcal{H}_{|D}|}  }{m} ]$
		
		由$|H_{|D}| \leq  {\textstyle \prod_{\mathcal{H}}}(m) $有
		
		$\mathcal{R}_m(\mathcal{H}) \leq E_D[\frac{\sqrt{m}\sqrt{2ln {\textstyle \prod_{\mathcal{H}}}(m) }  }{m} ] = \sqrt{\frac{2ln{\textstyle \prod_{\mathcal{H}}}(m) }{m} } $
		
		带入${\textstyle \prod_{\mathcal{H}}}(m) \leq \left ( \frac{em}{d}  \right ) ^d$即证: $\mathfrak{R}_m(\mathcal{H})\leq\sqrt{\frac{2d\ln\left(\frac{em}{d}\right)}{m}}~. $
		
		\textbf{(2)}
		
		设 $\mathcal{H}$ 为一族从 $\left \{ -1,+1 \right \} $ 中取值且 VC 维等于 d 的函数。那么,对于任意的 $\delta >0$,下面的不等式至少有 $1-\delta$的概率对假设 $h \in \mathcal{H}$成立
		
		$\mathcal{R}_m(h) \leq \hat{\mathcal{R}_m(h)} + \sqrt{\frac{2dlog\frac{em}{d} }{m} }+\sqrt{\frac{log\frac{1}{\delta} }{2m} }  $
		
		因此,这种情况下的泛化界满足
		$\mathcal{R}_m(h) \leq \hat{\mathcal{R}_m(h)} + \mathcal{O}(\sqrt{\frac{log(\frac{m}{d} )}{\frac{m}{d} } } )$
		
		它说明了$\frac{m}{d} $对于泛化的重要性。这个推论同样符合奥卡姆剃刀原则
		
		(由Sauer 引理,我们还能得到以下的 VC ,不通过Rademacher复杂度得到:
		$\mathcal{R}_m(h) \leq \hat{\mathcal{R}_m(h)} + \sqrt{\frac{8dlog\frac{2em}{d} +8log\frac{4}{\delta} }{m} } $)
		
		
	\end{myProof}
	
	\newpage
	
	\section{[50pts] Generalization Bound and Covering Numbers} 
	
	\noindent 设~$\mathcal{H}$~为一假设空间, 假设的定义域为~$\mathcal{X}$, 值域为$\mathcal{Y} \subset \mathbb{R}$. $\forall \epsilon > 0$, 可如下定义覆盖数 (Covering Number): 
	\begin{equation}
		\mathcal{N} ( \mathcal{H} , \epsilon ) 
		= \min \left\{ k \in \mathbb{N} | \exists \{ h_1 , \cdots , h_k \} \subset \mathcal{H} , \text{ s.t. } \forall h \in \mathcal{H} , \exists i \in [k] , \Vert h - h_i \Vert_\infty \leqslant \epsilon \right\} , 
	\end{equation}
	其中~$ \Vert h - h_i \Vert_\infty = \max_{x \in \mathcal{X}} | h(x) - h_i(x) | $. 覆盖数可以衡量一个假设空间的复杂度: 覆盖数越大, 意味着这一假设空间越复杂. 本题利用覆盖数证明了平方损失下的一个泛化界, 该结论也可以说明覆盖数可以衡量假设空间的复杂度. 令~$\mathcal{D}$~为~$\mathcal{X} \times \mathcal{Y}$~上的一个分布, 且有标记样本根据这一分布采样得到. 定义~$h \in \mathcal{H}$~的泛化误差为
	\begin{equation}
		R(h) = \mathbb{E}_{(x,y) \sim \mathcal{D}} \left[ (h(x)-y)^2 \right] . 
	\end{equation}
	训练集~$S = \left\{ (x_1,y_1) , \cdots , (x_m,y_m) \right\}$~上的经验误差为
	\begin{equation}
		\hat{R}_S(h) = \frac{1}{m} \sum_{i=1}^m (h(x_i)-y_i)^2 . 
	\end{equation}
	设~$\mathcal{H}$~是有界的, 即~$\exists M>0$, 使得~$ \forall (x,y) \in \mathcal{X} \times \mathcal{Y} , h \in \mathcal{H} , | h(x) - y | \leqslant M $. 
	\begin{enumerate}[(1)]
		\item \textbf{[15pts]} 令~$L_S = R(h) - \hat{R}_S(h)$, 试证明~$\forall h_1 , h_2 \in \mathcal{H} , S $, 
		\begin{equation}
			\left| L_S(h_1) - L_S(h_2) \right| \leqslant 4 M \Vert h_1 - h_2 \Vert_\infty . 
		\end{equation}
		\item \textbf{[15pts]} 设~$\mathcal{H}$~可以被~$k$~个子集~$\mathcal{B}_1 , \cdots , \mathcal{B}_k$~覆盖, 即~$\mathcal{H} = \mathcal{B}_1 \cup \cdots \cup \mathcal{B}_k $. 试证明~$\forall \epsilon > 0$, 
		\begin{equation}
			\mathop{\text{Pr}} \limits_{S \sim \mathcal{D}^m} \left[ \sup_{h \in \mathcal{H}} \left| L_S(h) \right| \geqslant \epsilon \right] 
			\leqslant \sum_{i=1}^k \mathop{\text{Pr}} \limits_{S \sim \mathcal{D}^m} \left[ \sup_{h \in \mathcal{B}_i} \left| L_S(h) \right| \geqslant \epsilon \right] . 
		\end{equation}
		\item \textbf{[20pts]} 令~$k = \mathcal{N} \left( \mathcal{H} , \frac{\epsilon}{8M} \right)$, $\mathcal{B}_1 , \cdots , \mathcal{B}_k$~为~$\mathcal{H}$~的覆盖, 其中~$\forall i \in [k]$, $\mathcal{B}_i$~的圆心为~$h_i$, 半径为~$\frac{\epsilon}{8M}$. 试证明
		\begin{equation}
			\mathop{\text{Pr}} \limits_{S \sim \mathcal{D}^m} \left[ \sup_{h \in \mathcal{B}_i} \left| L_S(h) \right| \geqslant \epsilon \right] 
			\leqslant \mathop{\text{Pr}} \limits_{S \sim \mathcal{D}^m} \left[ \left| L_S(h_i) \right| \geqslant \frac{\epsilon}{2} \right] . 
		\end{equation}
		并利用~Hoeffding~不等式证明
		\begin{equation}
			\mathop{\text{Pr}} \limits_{S \sim \mathcal{D}^m} \left[ \sup_{h \in \mathcal{H}} \left| R(h) - \hat{R}_S(h) \right| \geqslant \epsilon \right] 
			\leqslant \mathcal{N} \left( \mathcal{H} , \frac{\epsilon}{8M} \right) 2 \mathrm{e}^{ - \frac{m\epsilon^2}{2M^4} } . 
		\end{equation}
	\end{enumerate}
	
	\begin{myProof}~\\ 
		
		\textbf{(1)}
		
		由于$R(h) = \mathbb{E}_{(x,y) \sim \mathcal{D}} \left[ (h(x)-y)^2 \right] . $
		训练集~$S = \left\{ (x_1,y_1) , \cdots , (x_m,y_m) \right\}$~上的经验误差为$\hat{R}_S(h) = \frac{1}{m} \sum_{i=1}^m (h(x_i)-y_i)^2$
		
		$\therefore L_S = R(h) - \hat{R}_S(h) = \mathbb{E}_{(x,y) \sim \mathcal{D}} \left[ (h(x)-y)^2 \right] - \frac{1}{m} \sum_{i=1}^m (h(x_i)-y_i)^2$ 
		
		$\forall h_1 , h_2 \in \mathcal{H} , S , \left| L_S(h_1) - L_S(h_2) \right| = $
		
		$\left| \mathbb{E}_{(x,y) \sim \mathcal{D}} \left[ (h_1(x)-y)^2 \right] - \frac{1}{m} \sum_{i=1}^m (h_1(x_i)-y_i)^2 - \left( \mathbb{E}_{(x,y) \sim \mathcal{D}} \left[ (h_2(x)-y)^2 \right] - \frac{1}{m} \sum_{i=1}^m (h_2(x_i)-y_i)^2 \right) \right|$
		
		$\because \forall h_1 , h_2 \in \mathcal{H}$
		
		$\therefore \mathbb{E}_{(x,y) \sim \mathcal{D}} \left[ (h_1(x)-y)^2 \right] = \mathbb{E}_{(x,y) \sim \mathcal{D}} \left[ (h_2(x)-y)^2 \right]$
		
		$\frac{1}{m} \sum_{i=1}^m (h_2(x_i)-y_i)^2 - \frac{1}{m} \sum_{i=1}^m (h_1(x_i)-y_i)^2 = $
		
		$\frac{1}{m} \sum_{i=1}^m \left ( (h_2(x_i)-y_i)^2 - (h_1(x_i)-y_i)^2 \right )  = $
		
		
		$\frac{1}{m} \sum_{i=1}^m \left ( (h_2(x_i)+h_1(x_i)-2y_i) (h_2(x)-h_1(x)) \right ) $
		
		$\because \mathcal{H}$~是有界的, 即~$\exists M>0$, 使得~$ \forall (x,y) \in \mathcal{X} \times \mathcal{Y} , h \in \mathcal{H} , | h(x) - y | \leqslant M $y$
		
		$\therefore |(h_1(x_i)+h_2(x_i)-2y_i)^2|\leq |(2*max(h_1(x_i),h_2(x_i))-2y_i)^2| \leq 4M$
		
		并且显然,$|h_1(x_i) - h_2(x_i)| \leq \Vert h_1 - h_2 \Vert_\infty = \max_{x \in \mathcal{X}} | h_1(x) - h_2(x) |$
		
		$\therefore \left| L_S(h_1) - L_S(h_2) \right| = |\frac{1}{m} \sum_{i=1}^m (h_2(x_i)-y_i)^2 - \frac{1}{m} \sum_{i=1}^m (h_1(x_i)-y_i)^2| = $
		
		$|\frac{1}{m} \sum_{i=1}^m \left ( (h_2(x_i)+h_1(x_i)-2y_i) (h_2(x)-h_1(x)) \right )| =
		
		$\frac{1}{m} \sum_{i=1}^m |  (h_2(x_i)+h_1(x_i)-2y_i) (h_2(x)-h_1(x)) | \leq \frac{1}{m} \sum_{i=1}^m (4M  \Vert h_1 - h_2 \Vert_\infty )=4M  \Vert h_1 - h_2 \Vert_\infty$
		
		即证明$\forall h_1 , h_2 \in \mathcal{H} , S , \left| L_S(h_1) - L_S(h_2) \right| \leqslant 4 M \Vert h_1 - h_2 \Vert_\infty$
		
		\textbf{(2)}
		
		当k=1时,显然有$\mathcal{H} = B_1$
		
		所以$ \mathop{\text{Pr}} \limits_{S \sim \mathcal{D}^m} \left[ \sup_{h \in \mathcal{H}} \left| L_S(h) \right| \geqslant \epsilon \right] = \sum_{i=1}^1 \mathop{\text{Pr}} \limits_{S \sim \mathcal{D}^m} \left[ \sup_{h \in \mathcal{B}_i} \left| L_S(h) \right| \geqslant \epsilon \right]$题目要求显然成立
		
		假设k=n时,对任意$\mathcal{H}的n个子集B_1,B_2...B_n$,都有$ \mathop{\text{Pr}} \limits_{S \sim \mathcal{D}^m} \left[ \sup_{h \in \mathcal{H}} \left| L_S(h) \right| \geqslant \epsilon \right] = \sum_{i=1}^n \mathop{\text{Pr}} \limits_{S \sim \mathcal{D}^m} \left[ \sup_{h \in \mathcal{B}_i} \left| L_S(h) \right| \geqslant \epsilon \right]$成立
		
		则k=n+1时,会新增加一个$B_{n+1}$,由上可知该$B_{n+1}$可以是k=n的某一种子集划分情况下,某一个子集$B_p$的子集,因此可以有
		
		
		$ \sum_{i=1}^n+1 \mathop{\text{Pr}} \limits_{S \sim \mathcal{D}^m} \left[ \sup_{h \in \mathcal{B}_i} \left| L_S(h) \right| \geqslant \epsilon \right] = \sum_{i=1}^n \mathop{\text{Pr}} \limits_{S \sim \mathcal{D}^m} \left[ \sup_{h \in \mathcal{B}_i} \left| L_S(h) \right| \geqslant \epsilon \right] + \mathop{\text{Pr}} \limits_{S \sim \mathcal{D}^m} \left[ \sup_{h \in \mathcal{B}_{n+1}} \left| L_S(h) \right| \geqslant \epsilon \right]$
		
		$\therefore \mathop{\text{Pr}} \limits_{S \sim \mathcal{D}^m} \left[ \sup_{h \in \mathcal{H}} \left| L_S(h) \right| \geqslant \epsilon \right] \leq \sum_{i=1}^{n+1} \mathop{\text{Pr}} \limits_{S \sim \mathcal{D}^m} \left[ \sup_{h \in \mathcal{B}_i} \left| L_S(h) \right| \geqslant \epsilon \right]$成立
		
		由数学归纳法可知, 对任意k(k>0),有
		
		~$\mathcal{H} = \mathcal{B}_1 \cup \cdots \cup \mathcal{B}_k $. $\forall \epsilon > 0, $有$\mathop{\text{Pr}} \limits_{S \sim \mathcal{D}^m} \left[ \sup_{h \in \mathcal{H}} \left| L_S(h) \right| \geqslant \epsilon \right] \leqslant \sum_{i=1}^k \mathop{\text{Pr}} \limits_{S \sim \mathcal{D}^m} \left[ \sup_{h \in \mathcal{B}_i} \left| L_S(h) \right| \geqslant \epsilon \right] . 
		
		
		\textbf{(3)}
		
		$\because k = \mathcal{N} \left( \mathcal{H} , \frac{\epsilon}{8M} \right)$
		
		$\therefore k = \mathcal{N} ( \mathcal{H} , \frac{\epsilon }{8M} ) 
		= \min \left\{ k \in \mathbb{N} | \exists \{ h_1 , \cdots , h_k \} \subset \mathcal{H} , \text{ s.t. } \forall h \in \mathcal{H} , \exists i \in [k] , \Vert h - h_i \Vert_\infty \leqslant \frac{\epsilon }{8M}  \right\}$
		
		由于$\forall i \in [k]$, $\mathcal{B}_i$~的圆心为~$h_i$, 半径为~$\frac{\epsilon}{8M}$
		
		$\therefore \mathop{\text{Pr}} \limits_{S \sim \mathcal{D}^m} \left[ \left| L_S(h_i) \right| \geqslant \frac{\epsilon}{2} \right] = R(h_i) -  \hat{R}_S(h_i) = \mathbb{E}_{(x,y) \sim \mathcal{D}} \left[ (h_i(x)-y)^2 \right] - \frac{1}{m} \sum_{j=1}^m (h_i(x_j)-y_j)^2$
		
		根据ppt对引理4.1证明过程中的$Pr(sup_{h \in \mathcal{H}} |\hat{E}_D(h) - \hat{E}_{D'}(h) | \geq \frac{1}{2}\epsilon  ) \geq \frac{1}{2}Pr(sup_{h \in \mathcal{H}}|E(h) - \hat{E}_D(h)| > \epsilon )$
		
		其中D和$D'$都是从D独立同分采样的训练集
		
		$\mathop{\text{Pr}} \limits_{S \sim \mathcal{D}^m} \left[ \left| L_S(h_i) \right| \geqslant \frac{\epsilon}{2} \right] \geq E_{B_i \in Q}[E_{S_i\sim D^m}[\mathcal{I}(sup_{h \in \mathcal{H}}(|\hat{E}_{B_i}(h) - \hat{E}_{D'}(h)| \geq \frac{1}{2}\epsilon  )]]$
		
		$\geq E_{h_i\sim B_i} [\mathcal{I}(|\hat{E}_D(h_0)-E(h_0)|-||\hat{E}_{h_i}(h_0)-E(h_0)| \geq \frac{1}{2}\epsilon  )] \geq 1-Pr(|\hat{E}_{h_i}(h_0)-E(h_0)| > \frac{1}{2}\epsilon  )$
		
		由Chebyshev不等式(1.21)可得
		
		$Pr(|\hat{E}_{h_i}(h_0)-E(h_0)| > \frac{1}{2}\epsilon  ) \leq \frac{4(1-E(h_0))E(h_0)}{\epsilon ^2m} \leq \frac{1}{\epsilon ^2m}  $
		
		由于$\forall i \in [k]$, $\mathcal{B}_i$~的圆心为~$h_i$, 半径为~$\frac{\epsilon}{8M}$,于是可得
		$\mathop{\text{Pr}} \limits_{S \sim \mathcal{D}^m} \left[ \sup_{h \in \mathcal{B}_i} \left| L_S(h) \right| \geqslant \epsilon \right] 
		\leqslant \mathop{\text{Pr}} \limits_{S \sim \mathcal{D}^m} \left[ \left| L_S(h_i) \right| \geqslant \frac{\epsilon}{2} \right]$
		
		
		b. 利用~Hoeffding~不等式证明
		$\mathop{\text{Pr}} \limits_{S \sim \mathcal{D}^m} \left[ \sup_{h \in \mathcal{H}} \left| R(h) - \hat{R}_S(h) \right| \geqslant \epsilon \right] 
		\leqslant \mathcal{N} \left( \mathcal{H} , \frac{\epsilon}{8M} \right) 2 \mathrm{e}^{ - \frac{m\epsilon^2}{2M^4} }$
		
		由(2)$\mathop{\text{Pr}} \limits_{S \sim \mathcal{D}^m} \left[ \sup_{h \in \mathcal{H}} \left| L_S(h) \right| \geqslant \epsilon \right] 
		\leqslant \sum_{i=1}^k \mathop{\text{Pr}} \limits_{S \sim \mathcal{D}^m} \left[ \sup_{h \in \mathcal{B}_i} \left| L_S(h) \right| \geqslant \epsilon \right]$
		
		且$\mathop{\text{Pr}} \limits_{S \sim \mathcal{D}^m} \left[ \sup_{h \in \mathcal{H}} \left| R(h) - \hat{R}_S(h) \right| \geqslant \epsilon \right] = \mathop{\text{Pr}} \limits_{S \sim \mathcal{D}^m} \left[ \sup_{h \in \mathcal{H}} \left| L_S(h) \right| \geqslant \epsilon \right] $
		
		$\therefore \mathop{\text{Pr}} \limits_{S \sim \mathcal{D}^m} \left[ \sup_{h \in \mathcal{H}} \left| R(h) - \hat{R}_S(h) \right| \geqslant \epsilon \right] \leq \sum_{i=1}^k \mathop{\text{Pr}} \limits_{S \sim \mathcal{D}^m} \left[ \sup_{h \in \mathcal{B}_i} \left| R(h) - \hat{R}_S(h) \right| \geqslant \epsilon \right]$
		
		$\leq  \mathcal{N} \left( \mathcal{H} , \frac{\epsilon}{8M} \right) \mathop{\text{Pr}} \limits_{S \sim \mathcal{D}^m} \left[ \sup_{h \in \mathcal{B}_i} \left| R(h) - \hat{R}_S(h) \right| \geqslant \epsilon \right]$
		
		显然,由Hoeffding不等式,有:$\mathop{\text{Pr}} \limits_{S \sim \mathcal{D}^m} \left[ \sup_{h \in \mathcal{B}_i} \left| R(h) - \hat{R}_S(h) \right| \geqslant \epsilon \right] \leq  2 \mathrm{e}^{ - \frac{m\epsilon^2}{2M^4} }$
		
		$\therefore \mathop{\text{Pr}} \limits_{S \sim \mathcal{D}^m} \left[ \sup_{h \in \mathcal{H}} \left| R(h) - \hat{R}_S(h) \right| \geqslant \epsilon \right] \leq \mathcal{N} \left( \mathcal{H} , \frac{\epsilon}{8M} \right) \mathop{\text{Pr}} \limits_{S \sim \mathcal{D}^m} \left[ \sup_{h \in \mathcal{B}_i} \left| R(h) - \hat{R}_S(h) \right| \geqslant \epsilon \right] \leq \mathcal{N} \left( \mathcal{H} , \frac{\epsilon}{8M} \right) 2 \mathrm{e}^{ - \frac{m\epsilon^2}{2M^4} }$
		
		
		
		
		
		
		
	\end{myProof}
	
\end{document}
