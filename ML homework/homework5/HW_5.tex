\documentclass[a4paper,UTF8]{article}

\usepackage{amssymb}
\usepackage{amsmath}
\usepackage{amsthm}
\usepackage{bm}
\usepackage{color}
\usepackage{ctex}
\usepackage{enumerate}
\usepackage{epsfig}
\usepackage[margin=1.25in]{geometry}
\usepackage{graphicx}
\usepackage{hyperref}
\usepackage{lipsum}
\usepackage{mdframed}
\usepackage{pgfplots}
\usepackage{tcolorbox}

\setlength{\evensidemargin}{.25in}
\setlength{\textwidth}{6in}
\setlength{\topmargin}{-0.5in}
\setlength{\topmargin}{-0.5in}

%%%%%%%%%%%%%%%%%%此处用于设置页眉页脚%%%%%%%%%%%%%%%%%%
\usepackage{fancyhdr}
\usepackage{lastpage}
\usepackage{layout}
\footskip = 12pt 
\pagestyle{fancy}
\lhead{2022年春季}                    
\chead{机器学习理论研究导引}
\rhead{作业五}
\cfoot{\thepage}                                                
\renewcommand{\headrulewidth}{1pt} %页眉线宽, 设为0可以去页眉线
\setlength{\skip\footins}{0.5cm} %脚注与正文的距离           
\renewcommand{\footrulewidth}{0pt} %页脚线宽, 设为0可以去页脚线

\makeatletter %设置双线页眉                                        
\def\headrule{{\if@fancyplain\let\headrulewidth\plainheadrulewidth\fi%
		\hrule\@height 1.0pt \@width\headwidth\vskip1pt	%上面线为1pt粗  
		\hrule\@height 0.5pt\@width\headwidth %下面0.5pt粗            
		\vskip-2\headrulewidth\vskip-1pt} %两条线的距离1pt        
	\vspace{6mm}} %双线与下面正文之间的垂直间距              
\makeatother  

%%%%%%%%%%%%%%%%%%%%%%%%%%%%%%%%%%%%%%%%%%%%%%
\numberwithin{equation}{section}
\newtheorem*{myProof}{Proof}
\newtheorem*{mySol}{Solution}

\usepackage{multirow}

\pgfplotsset{compat=1.16} 

\begin{document}
	
	\title{机器学习理论研究导引\\
		作业五}
	\author{陈晟\, MG21330006} 
	\maketitle
	%%%%%%%% 注意: 使用XeLatex 编译可能会报错, 请使用 pdfLaTex 编译 %%%%%%%
	
	\section*{作业提交注意事项}
	
	\begin{tcolorbox}
		\begin{enumerate}
			\item[(1)] 本次作业提交截止时间为~\textcolor{red}{\textbf{2022/05/31  23:59:59}}, 截止时间后不再接收作业, 本次作业记零分; 
			\item[(2)] 作业提交方式:使用此~LaTex~模板书写解答,只需提交编译生成的~pdf~文件,将~pdf~文件上传到以下ftp服务器的指定位置:
			\newline 地址:sftp://210.28.132.67:22,用户名:mlt2022,密码:mltspring2022@nju
			\newline 文件夹位置:/C:/Users/mlt2022/hw\_submissions/hw5\_submission/  ;
			\item[(3)] pdf 文件命名方式:学号-姓名-作业号-v版本号, 例~ MG1900000-张三-5-v1;如果需要更改已提交的解答,请在截止时间之前提交新版本的解答,并将版本号加一;
			\item[(4)] 未按照要求提交作业,或~pdf~命名方式不正确,将会被扣除部分作业分数. 
		\end{enumerate}
	\end{tcolorbox}
	
	\newpage
	
	\section{[50pts] Consistent Surrogate Loss}
	
	\noindent 考虑对率函数$\phi(t) = \log \left( 1 + \mathrm{e}^{-t} \right)$, 回答并证明下述问题. 
	\begin{enumerate}
		\item \textbf{[15pts]} 试求解最优实值输出函数$f_\phi^*(\bm{x})$. 
		\item \textbf{[10pts]} 试求解最优实值输出函数对应的最优替代泛化风险$R_\phi^*$. 
		\item \textbf{[25pts]} 证明对率函数针对原$0/1$目标函数具有替代一致性. 
	\end{enumerate}
	
	\begin{myProof}~\\
		
		(1)对率替代函数:$ \phi(t) = log(1+e^{-t})$
		
		替代函数 $\phi$在数据分布 $\mathcal{D}$ 上的泛化风险定义为
		$$
		R_{\phi}(f)=\mathbb{E}_{(\boldsymbol{x}, y) \sim \mathcal{D}}[\phi(y f(x))],
		$$
		进一步定义最优替代泛化 风险为
		$$
		R_{\phi}^{*}=\min _{f}\left(R_{\phi}(f)\right) .
		$$
		
		由替代函数 $\phi$在数据分布 $\mathcal{D}$ 上的泛化风险定义为
		$$
		R_{\phi}(f)=\mathbb{E}_{(\boldsymbol{x}, y) \sim \mathcal{D}}[\phi(y f(x))],
		$$
		可得
		$$
		\begin{aligned}
			R_{\phi}(f) &=\mathbb{E}_{(\boldsymbol{x}, y) \sim \mathcal{D}}[\phi(y f(\boldsymbol{x}))] \\
			&=\mathbb{E}_{\boldsymbol{x} \sim \mathcal{D}_{\mathcal{X}}}[\eta(\boldsymbol{x}) \phi(f(\boldsymbol{x}))+(1-\eta(\boldsymbol{x})) \phi(-f(\boldsymbol{x}))]
		\end{aligned}
		$$
		
		进一步根据$$
		R_{\phi}^{*}=\min _{f}\left(R_{\phi}(f)\right) .
		$$可得
		$$
		R_{\phi}^{*}=\mathbb{E}_{\boldsymbol{x} \sim \mathcal{D}_{X}}\left[\min _{f(\boldsymbol{x}) \in \mathbb{R}}(\eta(\boldsymbol{x}) \phi(f(\boldsymbol{x}))+(1-\eta(\boldsymbol{x})) \phi(-f(\boldsymbol{x})))\right],
		$$
		从而得到替代函数的最优实值输出函数 $f_{\phi}^{*}(x)$ 为
		$$
		f_{\phi}^{*}(\boldsymbol{x}) \in \underset{f(\boldsymbol{x}) \in \mathbb{R}}{\arg \min }(\eta(\boldsymbol{x}) \phi(f(\boldsymbol{x}))+(1-\eta(\boldsymbol{x})) \phi(-f(\boldsymbol{x}))) .
		$$
		将对率替代函数:$ \phi(t) = log(1+e^{-t})$带入上式可得:
		最优实值输出函数 $f_{\phi}^{*}(x)$ 为
		$$
		f_{\phi}^{*}(\boldsymbol{x}) \in \underset{f(\boldsymbol{x}) \in \mathbb{R}}{\arg \min }(\eta(\boldsymbol{x}) log(1+e^{-f(x)})+(1-\eta(\boldsymbol{x})) log(1+e^{f(x)})) .
		$$
		假设$\eta(\boldsymbol{x})$为常值,对$(\eta(\boldsymbol{x}) log(1+e^{-f(x)})+(1-\eta(\boldsymbol{x})) log(1+e^{f(x)}))$中的$f(x)$求导可得:
		
		$\because ln(1+e^{-f(x)}) = \frac{-e^{-f(x)}}{1+e^{-f(x)}} $
		
		$ln(1+e^{f(x)}) = \frac{e^{f(x)}}{1+e^{f(x)}} $
		
		$\therefore (\eta(\boldsymbol{x}) log(1+e^{-f(x)})+(1-\eta(\boldsymbol{x})) log(1+e^{f(x)}))'$
		
		$=\frac{e^{-f(x)}\eta(x)}{1+e^{-f(x)}} +  \frac{e^{f(x)}(1-\eta(x))}{1+e^{f(x)}}$
		
		$=\frac{e^{-f(x)}\eta(x)}{1+e^{-f(x)}} + \frac{(1-\eta(x))}{1+e^{-f(x)}}$
		
		$=\frac{1-\eta(x)-e^{-f(x)}\eta(x)}{1+e^{-f(x)}}$
		
		当$1-\eta(x)-e^{-f(x)}\eta(x)=0$时, $f(x) = ln\frac{\eta(x)}{1-\eta(x)}$
		
		即$f_{\phi}^{*}(\boldsymbol{x})  = ln\frac{\eta(x)}{1-\eta(x)}$
		
		(2)
		
		由(1)中
		$$
		R_{\phi}^{*}=\mathbb{E}_{\boldsymbol{x} \sim \mathcal{D}_{X}}\left[\min _{f(\boldsymbol{x}) \in \mathbb{R}}(\eta(\boldsymbol{x}) \phi(f(\boldsymbol{x}))+(1-\eta(\boldsymbol{x})) \phi(-f(\boldsymbol{x})))\right],
		$$
		
		最优泛化风险$R_{\phi}^{*}=\min _{f}\left(R_{\phi}(f)\right) = \mathbb{E}_{\boldsymbol{x} \sim \mathcal{D}_{X}}\left[\min _{f(\boldsymbol{x}) \in \mathbb{R}}(\eta(\boldsymbol{x}) \phi(f_{\phi}^{*}(\boldsymbol{x})))+(1-\eta(\boldsymbol{x})) \phi(-f_{\phi}^{*}(\boldsymbol{x})))\right]$
		
		$=\mathbb{E}_{\boldsymbol{x} \sim \mathcal{D}_{X}}\left[(\eta(\boldsymbol{x}) \phi(ln\frac{\eta(x)}{1-\eta(x)}))+(1-\eta(\boldsymbol{x})) \phi(-ln\frac{\eta(x)}{1-\eta(x)}))\right]$
		
		$=\mathbb{E}_{\boldsymbol{x} \sim \mathcal{D}_{X}}\left[(\eta(\boldsymbol{x}) ln(1+e^{-ln\frac{\eta(x)}{1-\eta(x)}})+(1-\eta(\boldsymbol{x})) ln(1+e^{ln\frac{\eta(x)}{1-\eta(x)}})\right]
		$
		
		$=\mathbb{E}_{\boldsymbol{x} \sim \mathcal{D}_{X}}\left[(\eta(\boldsymbol{x}) ln(1+\frac{1-\eta(x)}{\eta(x)})+(1-\eta(\boldsymbol{x})) ln(1+\frac{\eta(x)}{1-\eta(x)})\right]$
		
		(3)
		替代函数一致性的充分条件 [Zhang, 2004b]:
		对替代函数 $\phi$, 若最优实值输出函数满足 $f_{\phi}^{*} \in \mathcal{F}^{*}$, 且存在常数 $c>0$ 和 $s \geqslant 1$ 使
		$$
		|\eta(\boldsymbol{x})-1 / 2|^{s} \leqslant c^{s}\left(\phi(0)-\eta(\boldsymbol{x}) \phi\left(f_{\phi}^{*}(\boldsymbol{x})\right)-(1-\eta(\boldsymbol{x})) \phi\left(-f_{\phi}^{*}(\boldsymbol{x})\right)\right),
		$$
		则替代函数 $\phi$ 具在一致性.
		
		上面定理的证明如下:
		对任意函数 $f$ 和样本 $x \in \mathcal{X}$, 记
		$$
		\begin{aligned}
			\Delta_{1}(\boldsymbol{x})=& \eta(\boldsymbol{x}) \mathbb{I}(f(\boldsymbol{x}) \leqslant 0) \\
			&+(1-\eta(\boldsymbol{x})) \mathbb{I}(f(\boldsymbol{x}) \geqslant 0)-\min \{\eta(\boldsymbol{x}), 1-\eta(\boldsymbol{x})\}
		\end{aligned}
		$$
		根据 $\begin{aligned} R(f) &=\mathbb{E}_{(x, y) \sim \mathcal{D}}[\mathbb{I}(y f(x) \leqslant 0)] \\ &=\mathbb{E}_{x \sim \mathcal{D}_{x}}[\eta(x) \mathbb{I}(f(x) \leqslant 0)+(1-\eta(x)) \mathbb{I}(f(x) \geqslant 0)] \end{aligned}$有
		$$
		R(f)-R^{*}=\mathbb{E}_{x \sim D_{X}}\left[\Delta_{1}(x)\right] .
		$$
		根据 $\eta(\boldsymbol{x})-1 / 2$ 和 $f(\boldsymbol{x})$ 的不同取值, 下面分五种情况讨论 $\Delta_{1}(\boldsymbol{x})$ :
		- 当 $\eta(x)>1 / 2$ 且 $f(x)>0$ 时, 有 $\Delta_{1}(x)=0$;
		- 当 $\eta(x)>1 / 2$ 且 $f(x) \leqslant 0$ 时, 有 $\Delta_{1}(x)=2 \eta(x)-1$;
		- 当 $\eta(x)<1 / 2$ 且 $f(x) \geqslant 0$ 时, 有 $\Delta_{1}(x)=1-2 \eta(x)$;
		- 当 $\eta(\boldsymbol{x})<1 / 2$ 且 $f(\boldsymbol{x})<0$ 时, 有 $\Delta_{1}(\boldsymbol{x})=0$;
		-当 $\eta(x)=1 / 2$ 时, 无论 $f(x)$ 取何值, 都有 $\Delta_{1}(x)=0$.
		综合上述五种䢸况可得
		$$
		\Delta_{1}(x)=2 \mathbb{I}((\eta(x)-1 / 2) f(x) \leqslant 0)|\eta(x)-1 / 2|
		$$
		
		代入之前得公式 有
		$$
		R(f)-R^{*}=2 \mathbb{E}_{(\eta(x)-1 / 2) f(x) \leqslant 0}[|\eta(x)-1 / 2|] .
		$$
		对 $s \geqslant 1$, 根据 Jensen 不等式 $(1.11)$ 有 $(\mathbb{E}[X])^{s} \leqslant \mathbb{E}\left[X^{s}\right]$, 于是有
		$$
		R(f)-R^{*} \leqslant 2 \sqrt[3]{\mathbb{E}_{(\eta(x)-1 / 2) f(x)} \xi 0\left[|\eta(x)-1 / 2|^{8}\right]} .
		$$
		根据$\begin{aligned} R_{\phi}(f) &=\mathbb{E}_{(\boldsymbol{x}, y) \sim \mathcal{D}}[\phi(y f(\boldsymbol{x}))] \\ &=\mathbb{E}_{\boldsymbol{x} \sim \mathcal{D}_{\boldsymbol{x}}}[\eta(\boldsymbol{x}) \phi(f(\boldsymbol{x}))+(1-\eta(\boldsymbol{x})) \phi(-f(\boldsymbol{x}))] . \end{aligned}$, 分别令
		$$
		\begin{aligned}
			&\Delta_{2}(\boldsymbol{x})=\eta(\boldsymbol{x}) \phi(f(\boldsymbol{x}))+(1-\eta(\boldsymbol{x})) \phi(-f(\boldsymbol{x})) \\
			&\Delta_{3}(\boldsymbol{x})=\eta(\boldsymbol{x}) \phi\left(f_{\phi}^{*}(\boldsymbol{x})\right)+(1-\eta(\boldsymbol{x})) \phi\left(-f_{\phi}^{*}(\boldsymbol{x})\right)
		\end{aligned}
		$$
		结合书中 $(6.40)$ 和定理 $6.1$ 中条件 (6.35) 可得
		$$
		R(f)-R^{*} \leqslant 2 c \sqrt[n]{\mathrm{E}_{(\eta(x)-1 / 2) f(x) \leqslant 0}\left[\phi(0)-\Delta_{3}(x)\right]}
		$$
		另一方面, 根据书中 $(6.32) \sim(6.34)$ 有
		$$
		R_{\phi}(f)-R_{\phi}^{*} \geqslant \mathbb{E}_{(\eta(x)-1 / 2) f(x) \leqslant 0}\left[\Delta_{2}(x)-\Delta_{3}(x)\right]
		$$
		设函数
		$$
		\Gamma(t)=\eta(\boldsymbol{x}) \phi(t)+(1-\eta(\boldsymbol{x})) \phi(-t),
		$$
		易知 $\Gamma(f(x))=\Delta_{2}(x)$ 和 $\Gamma(0)=\phi(0)$. 根据凸函数性质可知, 当 $\phi(t)$ 是凸函数 时 $\Gamma(t)$ 也是凸函数, 以及当 $0 \in[a, b]$ 时有
		$$
		\Gamma(0) \leqslant \max \{\Gamma(a), \Gamma(b)\}
		$$
		下面分三种情况讨论:
		- 若 $\eta(x)>1 / 2$, 由之前得式子可知 $f_{\phi}^{*}(x)>0$, 以及由 $(\eta(x)-1 / 2) f(x) \leqslant 0$ 可知 $f(x) \leqslant 0$. 因此 $0 \in\left[f(x), f_{\phi}^{*}(x)\right]$, 进一步有
		$$
		\phi(0)=\Gamma(0) \leqslant \max \left\{\Gamma(f(x)), \Gamma\left(f_{\phi}^{*}(\boldsymbol{x})\right)\right\}
		$$
		根据 $(6.33)$ 有 $\Gamma(f(x)) \geqslant \Gamma\left(f_{\phi}^{*}(\boldsymbol{x})\right)$, 于是得到 $\phi(0) \leqslant \Gamma(f(x))=\Delta_{2}(\boldsymbol{x})$.
		
		- 若 $\eta(x)<1 / 2$, 同理有 $f(x) \geqslant 0, f_{\phi}^{*}(x)<0$ 和 $0 \in\left[f_{\phi}^{*}(x), f(x)\right]$, 以及
		$$
		\phi(0)=\Gamma(0) \leqslant \max \left\{\Gamma(f(x)), \Gamma\left(f_{\phi}^{*}(x)\right)\right\}=\Gamma(f(x))=\Delta_{2}(x)
		$$
		- 若 $\eta(x)=1 / 2$, 对凸函数 $\phi$ 有
		$$
		\phi(0) \leqslant \phi(f(x)) / 2+\phi(-f(x)) / 2=\Delta_{2}(x)
		$$
		综合上述三种情况, 我们有
		$$
		\phi(0) \leqslant \Delta_{2}(x)
		$$
		根据之前得式子, 对任何函数 $f$ 有
		$$
		\begin{aligned}
			R(f)-R^{*} & \leqslant 2 c \sqrt[3]{\mathbb{E}_{(\eta(x)-1 / 2) f(x) \leqslant 0}\left[\phi(0)-\Delta_{3}(x)\right]} \\
			&=2 c \sqrt[s]{\mathbb{E}_{(\eta(x)-1 / 2) f(x) \leqslant 0}\left[\phi(0)-\Delta_{2}(x)+\Delta_{2}(\boldsymbol{x})-\Delta_{3}(\boldsymbol{x})\right]} \\
			& \leqslant 2 c \sqrt[s]{\mathbb{E}_{(\eta(x)-1 / 2) f(x) \leqslant 0}\left[\Delta_{2}(x)-\Delta_{3}(x)\right]} \\
			& \leqslant 2 c \sqrt[s]{R_{\phi}(f)-R_{\phi}^{*}}
		\end{aligned}
		$$
		对任何函数列 $\hat{f}_{1}, \hat{f}_{2}, \cdots, \hat{f}_{m}, \cdots$, 根据上式可知
		$$
		R\left(\hat{f}_{m}\right)-R^{*} \leqslant 2 c \sqrt{R_{\phi}\left(\hat{f}_{m}\right)-R_{\phi}^{*}},
		$$
		因此当 $R_{\phi}\left(\hat{f}_{m}\right) \rightarrow R_{\phi}^{*}$ 时有 $R\left(\hat{f}_{m}\right) \rightarrow R^{*}$ 成立, 定理得证.
		
		
		
		
		$f^{*} \in \mathcal{F}^{*}=\{f$ : 当 $\eta(x)=1 / 2$ 时 $f(x)$ 可以是任意的实数; 当 $\eta(x) \neq 1 / 2$ 时 $f(x)(\eta(x)-1 / 2)>0\}$.
		
		显然$f_{\phi}^{*} \in \mathcal{F}^{*}$成立,其次,看是否存在常数$c>0$ 和 $s \geqslant 1$ 使
		$$
		|\eta(\boldsymbol{x})-1 / 2|^{s} \leqslant c^{s}\left(\phi(0)-\eta(\boldsymbol{x}) \phi\left(f_{\phi}^{*}(\boldsymbol{x})\right)-(1-\eta(\boldsymbol{x})) \phi\left(-f_{\phi}^{*}(\boldsymbol{x})\right)\right),
		$$
		
		将$\phi(x) = \ln \left( 1 + \mathrm{e}^{-x} \right)$以及$f_{\phi}^{*}(\boldsymbol{x})  = ln\frac{\eta(x)}{1-\eta(x)}$带入上式得:
		
		$$
		|\eta(\boldsymbol{x})-1 / 2|^{s} \leqslant c^{s}\left(ln2-\eta(x)ln(ln\frac{\eta(x)}{1-\eta(x)}) -(1-\eta(\boldsymbol{x})) ln(ln\frac{\eta(x)}{1-\eta(x)})\right),
		$$
		
		$=c^{s}\left(ln2 + ln(ln\frac{1-\eta(x)}{\eta(x)})\right)= c^{s}\left(ln2|(ln\frac{1-\eta(x)}{\eta(x)})|\right)$
		
		可以发现后面一项当$\eta(x) = 1/2$时或$\eta(x) \ne  1/2$,都存在常数 $c>0$ 和 $s \geqslant 1$ 使上式成立,即证对率函数针对原$0/1$目标函数具有替代一致性
		
		
	\end{myProof}
	
	\newpage 
	
	\section{[50pts] Convergence Rate}
	
	\noindent 试分析下述情形下梯度下降算法 (课件第9页, 课本7.2.1节) 的收敛率. 
	\begin{itemize}
		\item 可行域$\mathcal{W}$是凸集. 
		\item 目标函数$f$是$\mathcal{W}$上可微的$\alpha$-强凸函数. 
		\item 目标函数在$\mathcal{W}$上是$l$-Lipschitz连续的, 即
		\begin{equation}
			\forall \bm{u} \in \mathcal{W} , \Vert \nabla f(\bm{u}) \Vert \leqslant l . 
		\end{equation}
		\item 梯度下降算法的步长为: 
		\begin{equation}
			\forall t , \eta_t = \frac{1}{\alpha t} . 
		\end{equation}
	\end{itemize}
	
	\begin{myProof}~\\ 
		
		(1)当可行域$\mathcal{W}$是凸集时
		
		若函数为凸函数,则其定义域为凸集,满足条件,其可以采用梯度下降法达到$O(1 / \sqrt{T})$的收敛率[Nesterov 2018]
		其基本流程如下:
		1.任意初始化$w_{1} \in \mathcal{W}$
		
		2.for $t=1, \ldots, T$ do
		
		3.    梯度下降:$w_{t+1}^{\prime}=w_{t}-\eta_{t} \nabla f\left(w_{t}\right)$
		
		4.    投影:$\boldsymbol{w}_{t+1}=\Pi_{\mathcal{W}}\left(\boldsymbol{w}_{t+1}^{\prime}\right)$
		
		5.end for
		
		6.返回$\overline{\boldsymbol{w}}_{T}=\frac{1}{T} \sum_{t=1}^{T} \boldsymbol{w}_{t}$
		
		(2)目标函数f是$\mathcal{W}$上可微的$\alpha$-强凸函数
		
		考虑目标函数f:$\mathcal{W} \mapsto \mathbb{R}$是$\alpha$-强凸函数,对$\alpha$-强凸函数f有下面性质,若$\boldsymbol{w}^{*}$为其最优解,对于任意$\boldsymbol{w} \in \mathcal{W}$有:
		
		$f(\boldsymbol{w})-f\left(\boldsymbol{w}^{*}\right) \geqslant \frac{\lambda}{2}\left\|\boldsymbol{w}-\boldsymbol{w}^{*}\right\|^{2}$
		
		此外,若梯度有上界l,则
		$\left\|w-\boldsymbol{w}^{*}\right\| \leqslant \frac{2 l}{\lambda}$
		
		$f(\boldsymbol{w})-f\left(\boldsymbol{w}^{*}\right) \leqslant \frac{2 l^{2}}{\lambda}$
		
		如果考虑强凸且光滑的函数f,即$\alpha$-强凸函数可微时,还满足以下的条件:
		
		$f\left(\boldsymbol{w}^{\prime}\right) \leqslant f(\boldsymbol{w})+\left\langle\nabla f(\boldsymbol{w}), \boldsymbol{w}^{\prime}-\boldsymbol{w}\right\rangle+\frac{\gamma}{2}\left\|\boldsymbol{w}^{\prime}-\boldsymbol{w}\right\|^{2}, \forall \boldsymbol{w}, \boldsymbol{w}^{\prime} \in \mathcal{W}$
		
		对可微的$\alpha$-强凸函数梯度下降时,其本值任然时进行梯度下降更新后再投影到可行域,对于其梯度下降算法有如下定理[Nesterov,2013]:
		
		若目标函数满足可微的$\alpha$-强凸函数,或$\alpha$-强凸函数且光滑,梯度下降取得了线性收敛率$O\left(\frac{1}{\beta^{T}}\right)$,其中$\beta > 1$
		
		证明:根据目标函数的性质以及更新公式
		
		$f\left(\boldsymbol{w}_{t+1}\right)$
		$\leqslant f\left(\boldsymbol{w}_{t}\right)+\left\langle\nabla f\left(\boldsymbol{w}_{t}\right), \boldsymbol{w}_{t+1}-\boldsymbol{w}_{t}\right\rangle+\frac{\gamma}{2}\left\|\boldsymbol{w}_{t+1}-\boldsymbol{w}_{t}\right\|^{2}$
		
		$=\min _{\boldsymbol{w} \in \mathcal{W}}\left(f\left(\boldsymbol{w}_{t}\right)+\left\langle\nabla f\left(\boldsymbol{w}_{t}\right), \boldsymbol{w}-\boldsymbol{w}_{t}\right\rangle+\frac{\gamma}{2}\left\|\boldsymbol{w}-\boldsymbol{w}_{t}\right\|^{2}\right)$
		
		$\leqslant \min _{\boldsymbol{w} \in \mathcal{W}}\left(f(\boldsymbol{w})-\frac{\lambda}{2}\left\|\boldsymbol{w}-\boldsymbol{w}_{t}\right\|^{2}+\frac{\gamma}{2}\left\|\boldsymbol{w}-\boldsymbol{w}_{t}\right\|^{2}\right)$
		
		$\leqslant \min _{\substack{w=\alpha w^{*}+(1-\alpha) \boldsymbol{w}_{t} \\ \alpha \in[0,1]}}\left(f(\boldsymbol{w})+\frac{\gamma-\lambda}{2}\left\|\boldsymbol{w}-\boldsymbol{w}_{t}\right\|^{2}\right)$
		
		$=\min _{\alpha \in[0,1]}\left(f\left(\alpha \boldsymbol{w}^{*}+(1-\alpha) \boldsymbol{w}_{t}\right)+\frac{\gamma-\lambda}{2}\left\|\alpha \boldsymbol{w}^{*}+(1-\alpha) \boldsymbol{w}_{t}-\boldsymbol{w}_{t}\right\|^{2}\right)$
		
		$\leqslant \min _{\alpha \in[0,1]}\left(\alpha f\left(\boldsymbol{w}^{*}\right)+(1-\alpha) f\left(\boldsymbol{w}_{t}\right)+\frac{\gamma-\lambda}{2} \alpha^{2}\left\|\boldsymbol{w}^{*}-\boldsymbol{w}_{t}\right\|^{2}\right)$
		
		$=\min _{\alpha \in[0,1]}\left(f\left(\boldsymbol{w}_{t}\right)-\alpha\left(f\left(\boldsymbol{w}_{t}\right)-f\left(\boldsymbol{w}^{*}\right)\right)+\frac{\gamma-\lambda}{2} \alpha^{2}\left\|\boldsymbol{w}^{*}-\boldsymbol{w}_{t}\right\|^{2}\right)$
		
		$\leqslant \min _{\alpha \in[0,1]}\left(f\left(\boldsymbol{w}_{t}\right)-\alpha\left(f\left(\boldsymbol{w}_{t}\right)-f\left(\boldsymbol{w}^{*}\right)\right)+\frac{\gamma-\lambda}{2} \frac{2}{\lambda} \alpha^{2}\left(f\left(\boldsymbol{w}_{t}\right)-f\left(\boldsymbol{w}^{*}\right)\right)\right)$
		
		$=\min _{\alpha \in[0,1]}\left(f\left(\boldsymbol{w}_{t}\right)+\left(\frac{\gamma-\lambda}{\lambda} \alpha^{2}-\alpha\right)\left(f\left(\boldsymbol{w}_{t}\right)-f\left(\boldsymbol{w}^{*}\right)\right)\right)$
		
		如果$\frac{\lambda}{2(\gamma-\lambda)} \geqslant 1$, 令 $\alpha=1$,则有$$
		f\left(\boldsymbol{w}_{t+1}\right)-f\left(\boldsymbol{w}^{*}\right) \leqslant \frac{\gamma-\lambda}{\lambda}\left(f\left(\boldsymbol{w}_{t}\right)-f\left(\boldsymbol{w}^{*}\right)\right) \leqslant \frac{1}{2}\left(f\left(\boldsymbol{w}_{t}\right)-f\left(\boldsymbol{w}^{*}\right)\right)
		$$
		如果 $\frac{\lambda}{2(\gamma-\lambda)}<1$, 令 $\alpha=\frac{\lambda}{2(\gamma-\lambda)}$, 则有
		
		$$
		\begin{aligned}
			f\left(\boldsymbol{w}_{t+1}\right)-f\left(\boldsymbol{w}^{*}\right) & \leqslant\left(1-\frac{\lambda}{4(\gamma-\lambda)}\right)\left(f\left(\boldsymbol{w}_{t}\right)-f\left(\boldsymbol{w}^{*}\right)\right) \\
			&=\frac{4 \gamma-5 \lambda}{4(\gamma-\lambda)}\left(f\left(\boldsymbol{w}_{t}\right)-f\left(\boldsymbol{w}^{*}\right)\right)
		\end{aligned}
		$$
		结合书中 $(7.20)$ 和 (7.21), 令
		$$
		\beta= \begin{cases}\frac{\lambda}{\gamma-\lambda}, & \frac{\lambda}{2(\gamma-\lambda)} \geqslant 1 \\ \frac{4(\gamma-\lambda)}{4 \gamma-5 \lambda}, & \frac{\lambda}{2(\gamma-\lambda)}<1\end{cases}
		$$
		那么下式总是成立
		$$
		f\left(\boldsymbol{w}_{t+1}\right)-f\left(\boldsymbol{w}^{*}\right) \leqslant \frac{1}{\beta}\left(f\left(\boldsymbol{w}_{t}\right)-f\left(\boldsymbol{w}^{*}\right)\right)
		$$
		将上式扩展可得
		$$
		f\left(\boldsymbol{w}_{T}\right)-f\left(\boldsymbol{w}^{*}\right) \leqslant \frac{1}{\beta^{T-1}}\left(f\left(\boldsymbol{w}_{1}\right)-f\left(\boldsymbol{w}^{*}\right)\right)=O\left(\frac{1}{\beta^{T}}\right)
		$$ 
		
		即证可微的$\alpha$-强凸函数,或$\alpha$-强凸函数且光滑,梯度下降取得了线性收敛率$O\left(\frac{1}{\beta^{T}}\right)$,其中$\beta > 1$
		
		(3)目标函数在$\mathcal{W}$上是l-Lipschitz连续的,即$\forall \boldsymbol{u} \in \mathcal{W},\|\nabla f(\boldsymbol{u})\| \leqslant l$
		
		梯度下降收玫率 若目标函数是 $l$-Lipschitz 连续函数, 且可行域 有界, 则采用固定步长梯度下降的收玫率为 $O\left(\frac{1}{\sqrt{T}}\right)$.
		证明 假设可行域 $\mathcal{W}$ 直径为 $\Gamma$, 并且目标函数满足 $l$-Lipschitz 连续, 即对 于任意 $\boldsymbol{u}, \boldsymbol{v} \in \mathcal{W}$,
		$$
		\|\boldsymbol{u}-\boldsymbol{v}\| \leqslant \Gamma,\|\nabla f(\boldsymbol{u})\| \leqslant l .
		$$
		为了简化分析, 考虑固定的步长 $\eta_{t}=\eta$.
		对于任意的 $\boldsymbol{w} \in \mathcal{W}$,
		$$
		f\left(\boldsymbol{w}_{t}\right)-f(\boldsymbol{w}) \leqslant\left\langle\nabla f\left(\boldsymbol{w}_{t}\right), \boldsymbol{w}_{t}-\boldsymbol{w}\right\rangle=\frac{1}{\eta}\left\langle\boldsymbol{w}_{t}-\boldsymbol{w}_{t+1}^{\prime}, \boldsymbol{w}_{t}-\boldsymbol{w}\right\rangle
		$$
		
		$$
		\begin{aligned}
			&=\frac{1}{2 \eta}\left(\left\|\boldsymbol{w}_{t}-\boldsymbol{w}\right\|^{2}-\left\|\boldsymbol{w}_{t+1}^{\prime}-\boldsymbol{w}\right\|^{2}+\left\|\boldsymbol{w}_{t}-\boldsymbol{w}_{t+1}^{\prime}\right\|^{2}\right) \\
			&=\frac{1}{2 \eta}\left(\left\|\boldsymbol{w}_{t}-\boldsymbol{w}\right\|^{2}-\left\|\boldsymbol{w}_{t+1}^{\prime}-\boldsymbol{w}\right\|^{2}\right)+\frac{\eta}{2}\left\|\nabla f\left(\boldsymbol{w}_{t}\right)\right\|^{2} \\
			&\leqslant \frac{1}{2 \eta}\left(\left\|\boldsymbol{w}_{t}-\boldsymbol{w}\right\|^{2}-\left\|\boldsymbol{w}_{t+1}-\boldsymbol{w}\right\|^{2}\right)+\frac{\eta}{2}\left\|\nabla f\left(\boldsymbol{w}_{t}\right)\right\|^{2}
		\end{aligned}
		$$
		最后一个不等号利用了凸集合投影操作的非扩展性质 [Nemirovski et al., 2009]:
		$$
		\left\|\Pi_{\mathcal{W}}(x)-\Pi_{\mathcal{W}}(z)\right\| \leqslant\|\boldsymbol{x}-\boldsymbol{z}\|, \quad \forall x, z .
		$$
		注意到目标函数满足 $l$-Lipschitz连续, 由上面的公式可得
		$$
		f\left(\boldsymbol{w}_{t}\right)-f(\boldsymbol{w}) \leqslant \frac{1}{2 \eta}\left(\left\|\boldsymbol{w}_{t}-\boldsymbol{w}\right\|^{2}-\left\|\boldsymbol{w}_{t+1}-\boldsymbol{w}\right\|^{2}\right)+\frac{\eta}{2} l^{2} .
		$$
		对上式从 $t=1$ 到 $T$ 求和, 有
		$$
		\begin{aligned}
			\sum_{t=1}^{T} f\left(\boldsymbol{w}_{t}\right)-T f(\boldsymbol{w}) & \leqslant \frac{1}{2 \eta}\left(\left\|\boldsymbol{w}_{1}-\boldsymbol{w}\right\|^{2}-\left\|\boldsymbol{w}_{T+1}-\boldsymbol{w}\right\|^{2}\right)+\frac{\eta T}{2} l^{2} \\
			& \leqslant \frac{1}{2 \eta}\left\|\boldsymbol{w}_{1}-\boldsymbol{w}\right\|^{2}+\frac{\eta T}{2} l^{2} \leqslant \frac{1}{2 \eta} \Gamma^{2}+\frac{\eta T}{2} l^{2}
		\end{aligned}
		$$
		最后, 依据 Jensen 不等式可得
		$$
		\begin{aligned}
			f\left(\overline{\boldsymbol{w}}_{T}\right)-f(\boldsymbol{w}) &=f\left(\frac{1}{T} \sum_{t=1}^{T} \boldsymbol{w}_{t}\right)-f(\boldsymbol{w}) \\
			& \leqslant \frac{1}{T} \sum_{t=1}^{T} f\left(\boldsymbol{w}_{t}\right)-f(\boldsymbol{w}) \leqslant \frac{\Gamma^{2}}{2 \eta T}+\frac{\eta l^{2}}{2}
		\end{aligned}
		$$
		因此,
		$$
		f\left(\overline{\boldsymbol{w}}_{T}\right)-\min _{w \in \mathcal{W}} f(\boldsymbol{w}) \leqslant \frac{\Gamma^{2}}{2 \eta T}+\frac{\eta l^{2}}{2}=\frac{l \Gamma}{\sqrt{T}}=O\left(\frac{1}{\sqrt{T}}\right)
		$$
		
		其中步长设置为$\eta=\Gamma /(l \sqrt{T})$定理得证
		
		
		(4) 当步长设置为变长$\forall t, \eta_{t}=\frac{1}{\alpha t}$时的收敛率
		
		以(3)中的条件更改收敛率进行推算
		
		对(3)中间步骤,从 $t=1$ 到 $T$ 求和, 有
		$$
		\begin{aligned}
			\sum_{t=1}^{T} f\left(\boldsymbol{w}_{t}\right)-T f(\boldsymbol{w}) & \leqslant \frac{1}{2 \eta}\left(\left\|\boldsymbol{w}_{1}-\boldsymbol{w}\right\|^{2}-\left\|\boldsymbol{w}_{T+1}-\boldsymbol{w}\right\|^{2}\right)+\frac{\eta T}{2} l^{2} \\
			& \leqslant \frac{1}{2 \eta}\left\|\boldsymbol{w}_{1}-\boldsymbol{w}\right\|^{2}+\frac{\eta T}{2} l^{2} \leqslant \frac{1}{2 \eta} \Gamma^{2}+\frac{\eta T}{2} l^{2}
		\end{aligned}
		$$
		
		将$\eta_{t}=\frac{1}{\alpha t}$带入上式可得:
		
		$f\left(\boldsymbol{w}_{t}\right)-f(\boldsymbol{w}) \leqslant \frac{1}{2 \alpha t}\left(\left\|\boldsymbol{w}_{t}-\boldsymbol{w}\right\|^{2}-\left\|\boldsymbol{w}_{t+1}-\boldsymbol{w}\right\|^{2}\right)+\frac{\alpha t}{2} l^{2}$
		
		$$
		\begin{aligned}
			\sum_{t=1}^{T} f\left(\boldsymbol{w}_{t}\right)-T f(\boldsymbol{w}) & \leqslant \frac{1}{2 \alpha }\left(\left\|\boldsymbol{w}_{1}-\boldsymbol{w}\right\|^{2}-\left\|\boldsymbol{w}_{T+1}-\boldsymbol{w}\right\|^{2}\right)+\frac{\alpha  T^2}{2} l^{2} \\
		\end{aligned}
		$$
		
		$$
		\leqslant \frac{1}{2 \alpha}\left\|\boldsymbol{w}_{1}-\boldsymbol{w}\right\|^{2}+\frac{\alpha  T^2}{2} l^{2} \leqslant \frac{1}{2 \alpha} \Gamma^{2}+\frac{\alpha  T^2}{2} l^{2}
		$$
		最后, 依据 Jensen 不等式可得
		$$
		\begin{aligned}
			f\left(\overline{\boldsymbol{w}}_{T}\right)-f(\boldsymbol{w}) &=f\left(\frac{1}{T} \sum_{t=1}^{T} \boldsymbol{w}_{t}\right)-f(\boldsymbol{w}) \\
			& \leqslant \frac{1}{T} \sum_{t=1}^{T} f\left(\boldsymbol{w}_{t}\right)-f(\boldsymbol{w}) \leqslant \frac{\Gamma^{2}}{2 \alpha T}+\frac{\alpha T l^{2}}{2}
		\end{aligned}
		$$
		因此,
		$$
		f\left(\overline{\boldsymbol{w}}_{T}\right)-\min _{\boldsymbol{w} \in \mathcal{W}} f(\boldsymbol{w}) \leqslant \frac{\Gamma^{2}}{2 \alpha T}+\frac{\alpha T l^{2}}{2}
		$$
		
		当$\frac{\alpha T l^{2}}{2}=\frac{\Gamma^{2}}{2 \\alpha T}$时, $\alpha = \frac{\Gamma}{Tl}$
		
		$f\left(\overline{\boldsymbol{w}}_{T}\right)-\min _{\boldsymbol{w} \in \mathcal{W}} f(\boldsymbol{w}) \leqslant l\Gamma = O(1)$
		
		即其收敛率变为常数级别了
		
		
		
		
	\end{myProof}
	
\end{document}
