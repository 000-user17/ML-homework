\documentclass[a4paper,UTF8]{article}

\usepackage{amssymb}
\usepackage{amsmath}
\usepackage{amsthm}
\usepackage{bm}
\usepackage{color}
\usepackage{ctex}
\usepackage{enumerate}
\usepackage{epsfig}
\usepackage[margin=1.25in]{geometry}
\usepackage{graphicx}
\usepackage{hyperref}
\usepackage{lipsum}
\usepackage{mdframed}
\usepackage{pgfplots}
\usepackage{tcolorbox}

\setlength{\evensidemargin}{.25in}
\setlength{\textwidth}{6in}
\setlength{\topmargin}{-0.5in}
\setlength{\topmargin}{-0.5in}

%%%%%%%%%%%%%%%%%%此处用于设置页眉页脚%%%%%%%%%%%%%%%%%%
\usepackage{fancyhdr}
\usepackage{lastpage}
\usepackage{layout}
\footskip = 12pt 
\pagestyle{fancy}
\lhead{2022年春季}                    
\chead{机器学习理论研究导引}
\rhead{作业四}
\cfoot{\thepage}                                                
\renewcommand{\headrulewidth}{1pt} %页眉线宽, 设为0可以去页眉线
\setlength{\skip\footins}{0.5cm} %脚注与正文的距离           
\renewcommand{\footrulewidth}{0pt} %页脚线宽, 设为0可以去页脚线

\makeatletter %设置双线页眉                                        
\def\headrule{{\if@fancyplain\let\headrulewidth\plainheadrulewidth\fi%
\hrule\@height 1.0pt \@width\headwidth\vskip1pt	%上面线为1pt粗  
\hrule\@height 0.5pt\@width\headwidth %下面0.5pt粗            
\vskip-2\headrulewidth\vskip-1pt} %两条线的距离1pt        
 \vspace{6mm}} %双线与下面正文之间的垂直间距              
\makeatother  

%%%%%%%%%%%%%%%%%%%%%%%%%%%%%%%%%%%%%%%%%%%%%%
\numberwithin{equation}{section}
\newtheorem*{myProof}{Proof}
\newtheorem*{mySol}{Solution}

\usepackage{multirow}

\def \x {\boldsymbol{x}}
\def \L {\mathfrak{L}}
\def \z {\boldsymbol{z}}
\def \w {\boldsymbol{w}}
\def \E {\mathbb{E}}
\def \D {\mathcal{D}}
\newcommand\norm[1]{\left\lVert #1 \right\rVert}

\pgfplotsset{compat=1.16} 

\begin{document}

\title{机器学习理论研究导引\\
作业四}
\author{你的姓名\, 你的学号} 
\maketitle
%%%%%%%% 注意: 使用XeLatex 编译可能会报错, 请使用 pdfLaTex 编译 %%%%%%%

\section*{作业提交注意事项}

\begin{tcolorbox}
    \begin{enumerate}
        \item[(1)] 本次作业提交截止时间为~\textcolor{red}{\textbf{2022/05/17  23:59:59}}, 截止时间后不再接收作业, 本次作业记零分; 
        \item[(2)] 作业提交方式:使用此~LaTex~模板书写解答,只需提交编译生成的~pdf~文件,将~pdf~文件上传到以下ftp服务器的指定位置:
        \newline 地址:sftp://210.28.132.67:22,用户名:mlt2022,密码:mltspring2022@nju
        \newline 文件夹位置:/C:/Users/mlt2022/hw\_submissions/hw4\_submission/  ;
        \item[(3)] pdf 文件命名方式:学号-姓名-作业号-v版本号, 例~ MG1900000-张三-4-v1;如果需要更改已提交的解答,请在截止时间之前提交新版本的解答,并将版本号加一;
        \item[(4)] 未按照要求提交作业,或~pdf~命名方式不正确,将会被扣除部分作业分数. 
    \end{enumerate}
\end{tcolorbox}

\newpage

\section{[50pts] Rethinking Stability of SVR} 
教材5.3.2节证明了支持向量回归具有替换样本$\beta$-均匀稳定性, 其中$\beta=\frac{2r^2}{\lambda m}$. 试给出更紧的界, 即$\beta=\frac{r^2}{\lambda m}$.

\begin{myProof}~\\ 

\end{myProof}

\newpage

\section{[50pts] Generalization and Stability} 

\noindent 给定分布$\mathcal{D}$, 对任意 $k \in[m]$, 数据集 $D\sim \mathcal{D}^{m}$ 和样本 $\boldsymbol{z} \in \mathcal{X} \times \mathcal{Y}$, 若算法 $\mathfrak{L}$ 满足
$$
\begin{array}{l}
	\left|\hat{R}\left(\mathfrak{L}_{D}\right)-\sum_{\boldsymbol{z}^{\prime} \in D^{k, \boldsymbol{z}}} \frac{\ell\left(\mathfrak{L}_{D^{k, \boldsymbol{z}}}, \boldsymbol{z}^{\prime}\right)}{m}\right| \leqslant \beta_{1} \\
	\left|R\left(\mathfrak{L}_{D}\right)-\mathbb{E}_{\boldsymbol{z} \sim \mathcal{D}}\left[\ell\left(\mathfrak{L}_{D^{k, \boldsymbol{z}}}, \boldsymbol{z}\right)\right]\right| \leqslant \beta_{2}
\end{array}
$$
试证明: 对任意 $\epsilon>0$ 有
$$
	P_{D \sim \mathcal{D}^{m}}\left(\left|R\left(\mathfrak{L}_{D}\right)-\hat{R}\left(\mathfrak{L}_{D}\right)\right| \geqslant \epsilon+\beta_{2}\right) \leqslant 2 \exp \left(\frac{-2 \epsilon^{2}}{m\left(\beta_{1}+2 \beta_{2}\right)^{2}}\right)
$$

\begin{myProof}~\\ 

\end{myProof}

\end{document}
